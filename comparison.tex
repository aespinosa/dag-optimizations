\documentclass{article}

\title{Comparison of data-aware scheduling and task clustering}

\begin{document}

\section{Introduction} % (fold)
\label{sec:introduction}
We compare workflow optimization techniques like task clustering to a data-aware
scheduler.  Both approaches seek to reduce the overall makespan of a DAG
workflow by reducing the amount of data movement in between jobs.

% section introduction (end)

\section{Related work} % (fold)
\label{sec:related_work}

Various work on task clustering.  Some people also call this process graph
partitioning.  Pegasus \cite{Singh2008} team used level-based and user-specified
task clustering schemes.  Their system assumes that the users know how to best
partition their tasks well.  Tanaka and Tatebe \cite{Tanaka2012} used
multi-constraint graph partitioning algorithms.  This technique was
traditionally used for the domain decomposition of finite-element meshes.

Data-aware scheduling people.  Thain et. al. \cite{Thain2003} defined datasets
into pipeline and batch on how they are used in grid workloads.  

% section related_work (end)

\end{document}

